\section{Problema 2}

\subsection{Definizione dei sottoproblemi}

Per la risoluzione del problema, definiamo i sottoproblemi come segue: per ogni posizione $(i,\ j)$ all'interno di una scacchiera, 
considerata come una matrice di dimensioni $n \times 3$, cerchiamo di determinare lo score massimo che possiamo ottenere posizionando 
una pedina in quella specifica casella.
\newline

In altre parole, per ogni cella della scacchiera rappresentata dalle coordinate $(i,\ j)$, vogliamo calcolare 
il punteggio massimo che possiamo ottenere posizionando una pedina in quella cella, tenendo conto delle possibili pedine 
già posizionate in altre celle della scacchiera. Questo processo di calcolo del punteggio massimo per ogni cella 
costituisce il nucleo dei sottoproblemi che affronteremo nell'algoritmo di programmazione dinamica.

\begin{center}
    $OPT[i,\ j]$ \bf{è lo score massimo posizionando la pedina nella casella} $(i,\ j)$.
\end{center}

\subsubsection*{Struttura dati}

Inizialmente abbiamo concepito $OPT$ come una matrice di dimensioni $n \times 3$, dove ogni posizione rappresentava una cella 
della scacchiera e conteneva il massimo punteggio ottenibile posizionando una pedina in quella cella. 
Tuttavia, nel processo di combinare i vari sottoproblemi, abbiamo riconosciuto la necessità di due ulteriori vettori, 
potenzialmente aggiuntivi di 2 righe rispetto alla matrice iniziale:

\begin{itemize}
    \item $OPT_{COMB}[1:n]$: Questo array contiene i valori \textit{combinati} (vedere paragrafo successivo) 
    delle colonne della matrice $OPT$. Ogni elemento $OPT_{COMB}[j]$ rappresenta il
    risultato ottenuto da una specifica casistica della formula di Bellman,
    applicata alla $j-esima$ colonna della matrice $OPT$.
    \item $OPT_{MAX}[1:n]$: Questo array memorizza il massimo tra i tre valori della $j-esima$
    colonna di $OPT$ e il valore corrispondente di $OPT_{COMB}$. In altre parole,
    per ogni indice $j$, $OPT_{MAX}[j]$ rappresenta il massimo tra i punteggi della $j-esima$
    colonna di $OPT$ e il valore combinato della stessa colonna rappresentato da $OPT_{COMB}[j]$.
\end{itemize}

In sintesi, questi due vettori aggiuntivi, $OPT_{COMB}$ e $OPT_{MAX}$, sono essenziali per gestire in modo efficiente i punteggi dei 
sottoproblemi e per calcolare il punteggio massimo complessivo in ciascun passaggio dell'algoritmo.

\subsection{Casistiche}

\setchessboard{
    boardfontsize=20pt, % Set the size of the chessboard pieces
    showmover=false, % Hide the mover symbol
    maxfield=c3, % Set the dimensions of the chessboard
    boardfontencoding=LSBC4,
    setfontcolors,
    showmover=false
}

Dopo aver definito la struttura dati e il contenuto di OPT, manca solo di comprendere come combinare efficacemente i sottoproblemi. 
Per evitare ridondanze nei dati raccolti in OPT, possiamo generalizzare i vari casi in una matrice $3 \times 3$, considerando che 
collezionando lo score delle pedine più lontane, rischieremmo di avere valori duplicati poiché sono già inclusi negli OPT più vicini.

Per capire come combinare i sottoproblemi abbiamo analizzato quello che è il principale limite sul posizionamento delle pedine, 
infatti inizialmente abbiamo pensato alle seguenti combinazioni.
\newline
Sia $i$ l'indice della $i-esima$ riga e sia $j$ l'indice della $j-esima$ colonna.\\
Siano le pedine nere i valori di $OPT$ già calcolati e sia la pedina bianca il valore da aggiungere ad $OPT$ da calcolare.

\subsubsection*{Casi banali}

Per analizzare i casi banali, supponiamo che le soluzioni delle colonne $2$ e $3$ sono state già calcolate, ovvero sono presenti nella matrice $OPT$ e nei vettori $OPT_{MAX}$ e $OPT_{COMB}$.

\begin{itemize}
    \item {
    Volendo collezionare lo score della casella $(i, j) = (3, 1)$ in cui è posizionata la pedina bianca, per come è definito il problema, nella collona a destra possiamo collezionare il valore delle celle in cui sono posizionate le pedine nere nelle seguenti posizioni:
    \newline
    \begin{enumerate*}[label={}]
        \item{
            \chessboard[
                zero=false, % numbering starts at zero
                labelbottomformat=\arabic{filelabel},
                setpieces={Pa3, pb2},
            ]
            $OPT[3, 1] = OPT[2, 2] + score[3, 1]$ 
        }
        \item{
            \chessboard[
                zero=false, % numbering starts at zero
                labelbottomformat=\arabic{filelabel},
                setpieces={Pa3, pb1},
            ]
            $OPT[3, 1] = OPT[1, 2] + score[3, 1]$ 
        }
    \end{enumerate*}
    }
    \newpage
    \item {
        Volendo collezionare lo score della casella $(i, j) = (2, 1)$ in cui è posizionata la pedina bianca, per come è definito il problema, nella collona a destra possiamo collezionare il valore delle celle in cui sono posizionate le pedine nere nelle seguenti posizioni:
    \newline
    \begin{enumerate*}[label={}]   
        \item{
            \chessboard[
                zero=false, % numbering starts at zero
                labelbottomformat=\arabic{filelabel},
                setpieces={Pa2, pb1},
            ]
            $OPT[2, 1] = OPT[1, 2] + score[2, 1]$ 
        }
        \item{
            \chessboard[
                zero=false, % numbering starts at zero
                labelbottomformat=\arabic{filelabel},
                setpieces={Pa2, pb3},
            ]
            $OPT[2, 1] = OPT[3, 2] + score[2, 1]$ 
        }
        \item{
            \chessboard[
                zero=false, % numbering starts at zero
                labelbottomformat=\arabic{filelabel},
                setpieces={Pa2, pb1, pb3},
            ]
            $OPT[2, 1] = OPT[1, 2] + OPT_{COMB}[2] + score[2, 1]$ 
        }
    \end{enumerate*}
    }

    \item {
        Volendo collezionare lo score della casella $(i, j) = (1, 1)$ in cui è posizionata la pedina bianca, per come è definito il problema, nella collona a destra possiamo collezionare il valore delle celle in cui sono posizionate le pedine nere nelle seguenti posizioni:
    \newline
    \begin{enumerate*}[label={}]   
        \item{
            \chessboard[
                zero=false, % numbering starts at zero
                labelbottomformat=\arabic{filelabel},
                setpieces={Pa1, pb3},
            ]
            $OPT[1, 1] = OPT[3, 2] + score[1, 1]$ 
        }
        \item{
            \chessboard[
                zero=false, % numbering starts at zero
                labelbottomformat=\arabic{filelabel},
                setpieces={Pa1, pb2},
            ]
            $OPT[1, 1] = OPT[2, 2] + score[1, 1]$ 
        }
    \end{enumerate*}
    }
\end{itemize}

\subsubsection*{Caso combinato}

Successivamente, abbiamo notato l'esistenza di un'ulteriore caso, che chiameremo \textit{caso combinato}, ovvero il caso in cui andiamo a prendere sia 
il valore nella riga $i = 1$ che quello nella riga $i = 3$, il cui risultato viene poi salvato all'interno dell'array $OPT_{COMB}[j]$ dove $j$ indica la colonna.

In questo caso vogliamo andare a collezionare lo score delle pedine in posizione $(i, j) = (3, 1)$ e $(1, 1)$.

\chessboard[
    zero=false, % numbering starts at zero
    labelbottomformat=\arabic{filelabel},
    setpieces={Pa1, Pa3, pb2},
]

$OPT_{COMB}[1] = OPT[2, 2] + score[3, 1] + score[1, 1]$

\subsubsection*{Caso in cui escludiamo una colonna}

Inoltre, analizzando l'istanza data, abbiamo notato la possibilità di dover saltare una colonna $k$, poiché il massimo score possibile 
può essere ottenuto solo combinando valori incompatibili presenti nelle colonne $k+1$ e $k-1$.

Per ogni cella $(i, 1)$ con $i\in\{1, 2, 3\}$ abbiamo le seguenti casistiche:

\begin{enumerate*}[label={}]   
    \item{
        \chessboard[
            zero=false, % numbering starts at zero
            labelbottomformat=\arabic{filelabel},
            setpieces={Pa3, pc3},
        ]
    }
    \item{
        \chessboard[
            zero=false, % numbering starts at zero
            labelbottomformat=\arabic{filelabel},
            setpieces={Pa3, pc2},
        ]
    }
    \item{
        \chessboard[
            zero=false, % numbering starts at zero
            labelbottomformat=\arabic{filelabel},
            setpieces={Pa3, pc1},
        ]
    }
    \item{
        \chessboard[
            zero=false, % numbering starts at zero
            labelbottomformat=\arabic{filelabel},
            setpieces={Pa3, pc1, pc3},
        ]
    }
\end{enumerate*}

\newpage
Inoltre bisogna tenere conto anche del caso in cui andiamo a combinare le celle, dunque otteniamo:

\begin{enumerate*}[label={}]   
    \item{
        \chessboard[
            zero=false, % numbering starts at zero
            labelbottomformat=\arabic{filelabel},
            setpieces={Pa3, Pa1, pc3},
        ]
    }
    \item{
        \chessboard[
            zero=false, % numbering starts at zero
            labelbottomformat=\arabic{filelabel},
            setpieces={Pa3, Pa1, pc2},
        ]
    }
    \item{
        \chessboard[
            zero=false, % numbering starts at zero
            labelbottomformat=\arabic{filelabel},
            setpieces={Pa3, Pa1, pc1},
        ]
    }
    \item{
        \chessboard[
            zero=false, % numbering starts at zero
            labelbottomformat=\arabic{filelabel},
            setpieces={Pa3, Pa1, pc1, pc3},
        ]
    }
\end{enumerate*}

Tuttavia, se manteniamo il massimo per ogni colonna in $OPT_{MAX}$, questi casi possono essere semplificati in uno solo, specificato poi nella formula di Bellman.  

Si osserva facilmente che escludere un numero maggiore di colonne non produce alcun guadagno.

\subsubsection*{Numero dei sottoproblemi}

Analizzando così le casistiche possiamo già determinare quale sia il numero di sottoproblemi:

\begin{itemize}
    \item Per $j = 1$ abbiamo 2 casi banali e il caso in cui si esclude una colonna.
    \item Per $j = 2$ abbiamo 3 casi banali e il caso in cui si esclude una colonna.
    \item Per $j = 3$ abbiamo 2 casi banali e il caso in cui si esclude una colonna.
    \item Per il combinato, abbiamo un caso banale e il caso in cui si esclude una colonna.
\end{itemize}

Osservando il numero di sottoproblemi per ogni colonna, notiamo che questo è uguale a 12. 
Moltiplicando questo numero per il numero totale di colonne, otteniamo un totale di sottoproblemi pari a $12n = O(n)$ nella dimensione dell'input. 

\subsection{Formula di Bellman}

\subsubsection*{Caso base}

Il nostro caso base è la colonna $n$ dove non abbiamo bisogno di combinare gli score con altri valori di OPT.

\begin{align*}
    \begin{cases}
        OPT[i, n] = score[i, n] \ \ \ \ \ \forall\ i = 1, 2, 3 \\
        OPT_{COMB}[n] = score[1,n] + score[3,n] \\
        OPT_{MAX}[n] = \max\{OPT[1,n],\ OPT[2,n],\ OPT[3,n],\ OPT_{COMB}[n]\} \\
    \end{cases}
\end{align*}

\subsubsection*{Casi generici}

\begin{align*}
    \begin{cases}
        OPT[1, i] = \max\{OPT[2, i + 1],\ OPT[3, i + 1],\ OPT_{MAX}[i + 2]\} + score[1, i] \\
        OPT[2, i] = \max\{OPT[1, i + 1],\ OPT[3, i + 1],\ OPT_{COMB}[i + 1],\ OPT_{MAX}[i + 2]\} + score[2, i] \\
        OPT[3, i] = \max\{OPT[1, i + 1],\ OPT[2, i + 1],\ OPT_{MAX}[i + 2]\} + score[3, i] \\
        OPT_{COMB}[i] = \max\{OPT[2, i + 1],\ OPT_{MAX}[i + 2]\} + score[1, i] + score[3, i] \\
        OPT_{MAX}[i] = \max\{OPT[1, i],\ OPT[2, i],\ OPT[3, i],\ OPT_{COMB}[i]\} \\
    \end{cases}
\end{align*}

\subsection{Pseudocodice}

\begin{algorithm}
    \caption{compute\_opt(scac)}
    \begin{algorithmic}[1]
    \Function{compute\_opt}{scac}
        \State Matrix $OPT$ $3 \times n$
        \State Array $OPT_{MAX}[n]$
        \State Array $OPT_{COMB}[n]$ 
        \State \Comment{Caso base}
        \For{$i = 1$ to $3$ do:}
            \State $OPT[i, n] = score[i, n]$
        \EndFor

        \State $OPT_{COMB}[n] = score[1,n] + score[3,n]$
        \State $OPT_{MAX}[n] = \max\{OPT[1,n],\ OPT[2,n],\ OPT[3,n],\ OPT_{COMB}[n]\}$
        \State \Comment{Fine caso base}


    \EndFunction
    \end{algorithmic}
\end{algorithm}

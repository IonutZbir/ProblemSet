\section{Problema 2}

\subsection{Definizione dei sottoproblemi}

Per la risoluzione del problema, definiamo i sottoproblemi come segue: per ogni posizione $(i,\ j)$ all'interno di una scacchiera, 
considerata come una matrice di dimensioni $n \times 3$, cerchiamo di determinare lo score massimo che possiamo ottenere posizionando 
una pedina in quella specifica casella.
\newline

In altre parole, per ogni cella della scacchiera rappresentata dalle coordinate $(i,\ j)$, vogliamo calcolare 
il punteggio massimo che possiamo ottenere posizionando una pedina in quella cella, tenendo conto delle possibili pedine 
già posizionate in altre celle della scacchiera. Questo processo di calcolo del punteggio massimo per ogni cella 
costituisce il nucleo dei sottoproblemi che affronteremo nell'algoritmo di programmazione dinamica.

\begin{center}
    $OPT[i,\ j]$ \bf{è lo score massimo posizionando la pedina nella casella} $(i,\ j)$.
\end{center}

\subsubsection*{Struttura dati}

Inizialmente abbiamo concepito $OPT$ come una matrice di dimensioni $n \times 3$, dove ogni posizione rappresentava una cella 
della scacchiera e conteneva il massimo punteggio ottenibile posizionando una pedina in quella cella. 
Tuttavia, nel processo di combinare i vari sottoproblemi, abbiamo riconosciuto la necessità di due ulteriori vettori, 
potenzialmente aggiuntivi di 2 righe rispetto alla matrice iniziale:

\begin{itemize}
    \item $OPT_{COMB}[1:n]$: Questo array contiene i valori "combinati" (vedere paragrafo successivo) delle colonne della matrice $OPT$. Ogni elemento $OPT_{COMB}[i]$ rappresenta il risultato ottenuto da una specifica casistica della formula di Bellman, applicata alla $i-esima$ colonna della matrice $OPT$.
    \item $OPT_{MAX}[1:n]$: Questo array memorizza il massimo tra i tre valori della $i-esima$ colonna di $OPT$ e il valore corrispondente di $OPT_{COMB}$. In altre parole, per ogni indice $i$, $OPT_{MAX}[i]$ rappresenta il massimo tra i punteggi della $i-esima$ colonna di $OPT$ e il valore combinato della stessa colonna rappresentato da $OPT_{COMB}[i]$.
\end{itemize}

In sintesi, questi due vettori aggiuntivi, $OPT_{COMB}$ e $OPT_{MAX}$, sono essenziali per gestire in modo efficiente i punteggi dei 
sottoproblemi e per calcolare il punteggio massimo complessivo in ciascun passaggio dell'algoritmo.

\subsubsection*{Casistiche}

\setchessboard{
    boardfontsize=20pt, % Set the size of the chessboard pieces
    showmover=false, % Hide the mover symbol
    maxfield=c3, % Set the dimensions of the chessboard
    boardfontencoding=LSBC4,
    setfontcolors,
    showmover=false
}

\begin{enumerate}
    \item{
        \chessboard[
            zero=false, % numbering starts at zero
            labelbottomformat=\arabic{filelabel},
            setpieces={pc3, Pc1},
        ]
        % descrizione
    }

    \item{
        \chessboard[
            zero=false, % numbering starts at zero
            labelbottomformat=\arabic{filelabel},
        ]
        % descrizione
    }

    \item{
        \chessboard[
            zero=false, % numbering starts at zero
            labelbottomformat=\arabic{filelabel},
        ]
        % descrizione
    }

    \item{
        \chessboard[
            zero=false, % numbering starts at zero
            labelbottomformat=\arabic{filelabel},
        ]
        % descrizione
    }

    \item{
        \chessboard[
            zero=false, % numbering starts at zero
            labelbottomformat=\arabic{filelabel},
        ]
        % descrizione
    }

    \item{
        \chessboard[
            zero=false, % numbering starts at zero
            labelbottomformat=\arabic{filelabel},
        ]
        % descrizione
    }
\end{enumerate}

